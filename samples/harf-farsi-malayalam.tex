\documentclass{article}

\usepackage[bidi=basic]{babel}

\babelprovide[main, import]{persian}
\babelprovide[import]{malayalam}

\babelfont{rm}[Renderer=Harfbuzz]{Amiri}
\babelfont[malayalam]{rm}[Renderer=Harfbuzz]{FreeSerif}

\babeladjust{ bidi.mirroring = off } % Harf does mirroring

\begin{document}

حرف‌باز (به انگلیسی: HarfBuzz) یک کتاب‌خانه توسعه نرم‌افزار برای
شکل‌دهی به متون یونیکد است. حرف‌باز دو نسخه دارد که به حرف‌باز قدیم و
حرف‌باز جدید معروف است. حرف‌باز قدیم دیگر توسعه داده نمی‌شود. نسخه قدیم
تنها از فونت‌های نوع اوپن‌فونتز پشتیبانی می‌کرد، در حالی که نسخه جدید
از انواع مختلف تکنولوژی‌های مرتبط با فونت‌ها پشتیبانی می‌کند.

\medskip
\small

مشارکت‌کنندگان ویکی‌پدیا. «HarfBuzz». \textit{در دانشنامهٔ ویکی‌پدیای انگلیسی،}
بازبینی‌شده در ۲ شهریور ۱۳۹۲.

\normalsize
\medskip
\selectlanguage{malayalam}

അക്ഷരങ്ങൾ രൂപപ്പെടുത്താനുപയോഗിക്കുന്ന ഒരു സോഫ്റ്റ്‍വെയർ ലൈബ്രറിയാണ്
ഹാർഫ്ബസ്. യുണികോഡ് അക്ഷരങ്ങളെ ഗ്ലിഫുകളായി രൂപപ്പെടുത്തുന്നതിനും അവയുടെ
സ്ഥാനം നിർണ്ണയിക്കുന്നതിനുമാണ് ഹാർഫ്ബസ് ഉപയോഗിക്കുന്നത്. വിവിധ
സാങ്കേതിവിദ്യയിലുള്ള ഫോണ്ടുകൾ പ്രദർശിപ്പിക്കാൻ പുതിയ ഹാർഫ്ബസ്
ഉപയോഗിക്കുന്നു എന്നാൽ പഴയ് ഹാർഫ്ബസ് ഓപ്പൺടൈപ്പ് ഫോണ്ടുകൾ മാത്രമേ
പിൻതുണക്കുകയുള്ളൂ.

\medskip

Typeset with luahblatex. Requires luaotfload 3.11.

\end{document}