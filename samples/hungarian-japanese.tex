\documentclass{book}

\usepackage
  % [layout=sectioning]
  {babel}

\usepackage{hyperref}

\babelprovide[
  import, main,
  % labels = map, labels/arabic.map = informal
]{japanese}
\babelprovide[import]{hungarian}

\babelfont{rm}{IPAexMincho}
\babelfont[hungarian]{rm}{FreeSerif}

\begin{document}

\tableofcontents

\chapter{日本}

日本語を通用する日本人が国民の大半を占める。自然地理的には、ユーラシア大陸の東縁に位置しており、環太平洋火山帯を構成する。島嶼国であり、領土の四方を海に囲まれているため地続きの国境は存在しない。日本列島は、離島も含めて6,848(時代により変動)の島を有する。気候区分は、北は亜寒帯から南は亜熱帯まで様々な気候区分に属している。

\selectlanguage{hungarian}

\chapter{Magyarország}

A finnugor nyelvcsalád legnagyobb lélekszámú és legnyugatibb csoportját
képező, vándorlásaik során török és keleti szláv népekkel érintkező, a
Kárpát-medencében való letelepedésüket követően számos etnikai elemet
és kulturális jelenséget magába olvasztó, illetve átvevő magyarság
etnikai-táji szerkezete más európai népekéhez képest egységes.

\selectlanguage{japanese}

\appendix

\chapter{日本}

日本語を通用する日本人が国民の大半を占める。自然地理的には、ユーラシア大陸の東縁に位置しており、環太平洋火山帯を構成する。島嶼国であり、領土の四方を海に囲まれているため地続きの国境は存在しない。日本列島は、離島も含めて6,848(時代により変動)の島を有する。気候区分は、北は亜寒帯から南は亜熱帯まで様々な気候区分に属している。

\selectlanguage{hungarian}

\chapter{Magyarország}

A finnugor nyelvcsalád legnagyobb lélekszámú és legnyugatibb csoportját
képező, vándorlásaik során török és keleti szláv népekkel érintkező, a
Kárpát-medencében való letelepedésüket követően számos etnikai elemet
és kulturális jelenséget magába olvasztó, illetve átvevő magyarság
etnikai-táji szerkezete más európai népekéhez képest egységes.

\end{document}