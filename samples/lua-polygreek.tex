\documentclass{article}

\usepackage{babel}

\usepackage{multicol}

\babelprovide[import,main]{polytonicgreek}
\babelfont{rm}{Old Standard}

\begin{document}

\section*{Φορτωμένα κόκκαλα}

\textit{Αλέξανδρος Παπαδιαμάντης}\\
1907\\
From https://el.wikisource.org/wiki/Φορτωμένα\_κόκκαλα

\begin{multicols}{3}

Ἀνεβαίνομεν τὸ βουνὸν, πεζοὶ, μὲ τὸ γαϊδουράκι φορτωμένον, ὁ παπ’
Ἀνδρέας, ὁ καλός μας εὐχέτης, κι’ ὁ μακαρίτης ὁ Λαμιαῖος, κι’ ἐγὼ, κι’
ὁ Ἀλέκος τὸ Φωτάκι, ὁ μικρὸς καὶ πρόθυμος φίλος μας. Εἰς ὅλα ἦτον
πάντοτε ἕτοιμος νὰ τρέχῃ ἀκούραστος, ὅ,τι ἤθελες τὸν διατάξει. Νὰ πάῃ
στὸ χωριὸ, διὰ θέλημα, δυὸ ὧρες δρόμον, καὶ πάλι, φορτωμένος, ὀπίσω νὰ
ἔλθῃ· νὰ σκουπίσῃ ὅλον τὸ ἐξωκκλήσι, καὶ τὸν αὐλόγυρον, καὶ τὰ κελλιὰ,
μὲ πρόχειρον σκούπαν ἀπὸ σπαρτίνες καὶ θάμνους· νὰ τρέξῃ κάτω στὸν
αἰγιαλόν, διὰ νὰ μᾶς φέρῃ πεταλίδες, καὶ κοχύλια, καὶ πετροκάβουρα, διὰ
τὸ ὀρεκτικὸν δεῖπνόν μας καὶ νὰ γυρίσῃ μετὰ μίαν ὥραν μὲ μίαν ποδιὰν
γεμάτην· εἶτα ν’ ἀνάψῃ φωτιάν, νὰ ψήσῃ, νὰ μαγειρεύσῃ ὅλα τὰ ἐδέσματα·
καὶ νὰ ἔχῃ τὴν ἐπιστασίαν τοῦ παγουριοῦ καὶ τῆς φλάσκας, διὰ νὰ
εὑρίσκωνται δροσερὰ εἰς τὸ ρεῦμα, ἀκριβῶς ὑπὸ τὴν βρύσην· εἰς ὅλα ἦτον
μονάκριβος.

Ἕκτος μᾶς εἶχεν ἀκολουθήσει ὁ σκύλος τοῦ Σταμάτη τοῦ Ἀλεξανδράκη,
σκύλος προωρισμένος ὀψέποτε νὰ μένῃ ἀδέσποτος. Ὁ ἄτυχος καὶ κακοκέφαλος
φίλος μας, ὁ Σταμάτης, ἀφοῦ ἐμάλωσε μὲ ὅλους τοὺς συγγενεῖς καὶ τοὺς
φίλους του, καὶ σχεδὸν μὲ ὅλον τὸν κόσμον, ἤρχισε νὰ πάσχῃ ἀπὸ
περιοδικὰς ἀφανείας, ὁποῦ ἦσαν τὰ προανακρούσματα τῆς ὁριστικῆς
ἐξαφανίσεώς του ἀπὸ τὸν μάταιον κόσμον. Πότε ἐκρύπτετο, ὡς ἔλεγαν, εἰς
μίαν ἐρημικὴν σπηλιὰν, πότε ἐπήγαινε νὰ μείνῃ ὀλίγας ἡμέρας εἰς τὸ
Μοναστήρι, πότε ἐταξίδευεν, ἄγνωστον ποῦ· καὶ ὅλας αὐτὰς τὰς φορὰς, τὸν
ἄτυχον Σαψώνην, τὸν ἄφηνεν εἰς τὸ ἔλεος τοῦ Θεοῦ καὶ τῶν ἀνθρώπων τῆς
ἀγορᾶς, ἂν θὰ εὐαρεστοῦντο ποτὲ νὰ τοῦ ρίψουν ἓν ξηροκόμματον. Συχνὰ ὁ
Γιωργὸς ὁ Λαφκιώτης ὁ ἰδιοκτήτης τοῦ ἀρχαιοπρεποῦς καὶ ἀναλλοιώτου
καφενείου εἰς τὴν παραθαλασσίαν, ἂν καὶ τοῦ εἶχε φάγῃ ὀλίγας
ἑκατοντάδας δραχμῶν, καλῇ τῇ πίστει, ὁ ἀφέντης τοῦ Σαψώνη, ᾤκτειρε τὸ
ἄκακον θρέμμα, καὶ τοῦ ἔρριπτεν ὀλίγα κόκκαλα. Ἐμὲ, ἀφοῦ μὲ
κατεσκόπευεν ὁ Σαψώνης, ἀπὸ καφενεῖον εἰς ὕπαιθρον, καὶ ἀπὸ κιόσκι εἰς
τένταν, ἐπὶ τῆς προκυμαίας, τέλος, μὲ ἠκολούθει ὁριστικῶς εἰς τὴν
οἰκίαν ὅπου ἔπρεπε νὰ τοῦ ρίψω τι ἐκ τοῦ ὑστερήματος.

Ἀλλὰ καὶ εἰς τὰ καλά του ὅταν ἦτο ὁ Σταμάτης, διὰ τὸν περιπαθῶς
ἀφωσιωμένον σκύλον δὲν εἶχε λάβῃ ἄλλην πρόνοιαν, εἰμὴ νὰ τὸν ρίπτῃ
αἰφνιδίως εἰς τὸ κῦμα, γαυγίζοντα καὶ μὴ θέλοντα, διὰ νὰ κολυμβᾷ. Τὸν
εἶχεν ἀφήσει ἀκούρευτον καὶ βαθύτριχον ἀπὸ χρόνων πολλῶν. Ἦτο πολὺ
μαλλιαρὸς σκύλος. Τὴν φορὰν αὐτὴν, ἀφοῦ μᾶς ἐμυρίσθη πῶς ἡτοιμαζόμεθα
δι’ ἐκδρομὴν, εἶχε δείξει ἀνήσυχον περιέργειαν, ὅταν ἐφορτώνετο τὸ
ὀνάριον, καὶ εἶχε πλησιάσει νὰ ὀσφρανθῇ τὶ περιεῖχεν ὁ σάκκος ἐκεῖνος,
τὸν ὁποῖον ὁ Ἀλέκος εἶχε περιδέσει καὶ φορτώσει περὶ τὸ σάγμα, εἰς τὴν
ἀριστερὰν πλευρὰν τοῦ ζώου. Δεξιὰ εἶχε φορτωθῇ ὁ μάρσιππος μὲ τὰ ἱερὰ
τοῦ Παπᾶ, εἷς κάλαθος μὲ τρόφιμα, καὶ μία φλάσκα μὲ οἶνον. Εἶτα, μᾶς
ἠκολούθησε μὲ βῆμα, αὐτόκλητος.
\end{multicols}

\end{document}